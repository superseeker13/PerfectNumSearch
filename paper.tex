\\\documentclass[12pt]{article}
\usepackage{amssymb, amsmath, amsthm}
\renewcommand{\baselinestretch}{1.2}
\usepackage[utf8]{inputenc}
\usepackage[english]{babel}
 
\newtheorem{theorem}{Theorem}

\begin{document}
\begin{center}
	\Large
	The search for an odd perfect number
\end{center}

\begin{flushright}
	Edward C.\\
	\vspace{2 mm}
	Maura M.\\
	\vspace{2 mm}
	Thomas R.\\
\end{flushright}

\paragraph*{Introduction}
	A non-negative integer n is perfect if it can be expressed as the sum of its proper divisors. This gives following equation 
	$$\text{n is perfect if } n = \sigma(n) = \sum_{}{}\{k \in \mathbb{Z}^+ : k|n\}$$  Where $\sigma_x (n)$ 
	denotes the sum of positive divisors function.  The study of such numbers dates back to at least the 
	time of Euclid. In Book IX Proposition 36 his elements he details a elementary method for finding even
	perfect numbers.  This method was later labeled the Euclid–Euler theorem
	after Euler proved that the method produced even perfect numbers.
	
	The following is a proof of the afterfore mentioned theorem.
	
	\begin{theorem}
		An even natural number is perfect iff it of the form $$2^{p-1} (2^{p} - 1)$$
	\end{theorem}
	
	\begin{proof}
		$(\Rightarrow)$
			Show an even natural number is perfect if it is of the form  $$2^{p-1} (2^{p} - 1).$$
			Let $p \in \mathbb{N}$ s.t. $2^{p}-1$ is prime and $\sigma_x (n)$, as above, denote the sum of positive divisors function.
			It follows that $$\sigma (2^{p-1}(2^{p}-1))=\sigma (2^{p-1})\sigma (2^{p}-1) \newline =>
			\sigma (2^{p-1}(2^{p}-1))=\sigma (2^{p-1})\sigma (2^{p}-1)$$
			Where $\sigma (2^{p-1}) = \sum{}{}(1 + 2 + 4 + ... + 2^{p-1})$ is a geometric series.\newline
			Thus $\sigma (2^{p-1}) = 2^{p}-1)$ and $\sigma (2^{p}-1) = 2^{p}$ since $2^{p}-1$ is prime.
		
		$(\Leftarrow)$
			Show if an even natural number is of the form  $$2^{p-1} (2^{p} - 1)$$ then it is perfect.
			Suppose $n \in \mathbb{N}$ s.t. $$n = 2^{p-1} (2^{p} - 1)$$ for $p \in \mathbb{N}$
			Show n is perfect.
			Cosider the partial factorization of $n = 2^{x}y$ where y is odd.
			If $2^{x}$ is perfect then $2^{x}y = \sigma (2^{x}y) = $
 
	\end{proof}

\paragraph*{History of Perfect Numbers}
The smallest positive perfect number is 6. The proper divisors of 6 are 1, 2, and 3. Notice, $$1+2+3=6$$
The next perfect number is 28, whose proper divisors are 1, 2, 4, 7, and 14.
$$1+2+4+7+14=28$$
The next 6 perfect numbers are
 $$\begin{array}{r}
496\\
8128\\
33550336\\
8589869056\\
137438691328\\
2305843008139952128
\end{array}$$
To date, there are 51 known perfect numbers, with the most recent one being found in 2018. It is 49,724,095 digits long.
\paragraph*{Odd Perfect Numbers}
All known perfect numbers are even. In 1638, René Descartes wrote a letter to Marin Mersenne in which he discussed the possible existence of an odd perfect number. Before this, most assumed that no odd perfect numbers existed, but none could prove it.$^{[3]}$ \\
In 2012, Ochem and Rao showed there are no perfect numbers less than $10^{1500}$.$^{[1]}$

\paragraph*{Conclusion}

\paragraph*{Acknowledgment}


\begin{thebibliography}{5}
\bibitem{lowerbound} 
\textit{Ochem, P. and Rao, M. "Odd Perfect Numbers Are Greater than $10^{1500}$."} Math. Comput. 81, 1869-1877, 2012.
\bibitem{elements}
\textit{“Proposition 36.” Euclid's Elements, Book IX, Proposition 36,} https://mathcs.clarku.edu/~djoyce/java/elements/bookIX/propIX36.html.
\bibitem{DicksonPN}
\textit{Dickson, Leonard Eugene. History of the Theory of Numbers. Dover Publications, 2005.}
\end{thebibliography}

\end{document}
