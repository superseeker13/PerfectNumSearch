\documentclass[12pt]{article}
\usepackage{amssymb, amsmath}
\renewcommand{\baselinestretch}{1.2}

\begin{document}
\begin{center}
\Large
The search for an odd perfect number
\end{center}

\begin{flushright}
Edward C.\\
\vspace{2 mm}
Maria M.\\
\vspace{2 mm}
Thomas R.\\
\end{flushright}

\paragraph*{Introduction}
A non-negative integer n is perfect if it can be expressed as the sum of its proper divisors.  
That is $$\text{n is perfect if }\sum_{}{}\{k \in \mathbb{Z}^+ : k \equiv 0 \mod n\} = n$$
The study of such numbers dates back to at least the time of Euclid. In Book IX Proposition 36 his elements he details a elementary method for finding even perfect numbers know today as Euclid’s Perfect Number Theorem.
\paragraph*{Conclusion}

\paragraph*{Acknowledgment}


\begin{thebibliography}{5}
\bibitem{lowerbound} 
\textit{Ochem, P. and Rao, M. "Odd Perfect Numbers Are Greater than 10\^(15000)."} Math. Comput. 81, 1869-1877, 2012.
\bibitem{elements}
\textit{“Proposition 36.” Euclid's Elements, Book IX, Proposition 36,} https://mathcs.clarku.edu/~djoyce/java/elements/bookIX/propIX36.html.
\end{thebibliography}

\end{document}
