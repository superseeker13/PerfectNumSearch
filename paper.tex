\documentclass[12pt]{article}
\usepackage{amssymb, amsmath, amsthm}
\renewcommand{\baselinestretch}{1.2}
\usepackage[utf8]{inputenc}
\usepackage[english]{babel}
 
\newtheorem{theorem}{Theorem}

\begin{document}
\begin{center}
	\Large
	The search for an odd perfect number
\end{center}

\begin{flushright}
	Edward C.\\
	\vspace{2 mm}
	Maria M.\\
	\vspace{2 mm}
	Thomas R.\\
\end{flushright}

\paragraph*{Introduction}
	A non-negative integer n is perfect if it can be expressed as the sum of its proper divisors. This gives following equation 
	$$\text{n is perfect if } n = \sigma(n) = \sum_{}{}\{k \in \mathbb{Z}^+ : k|n\}$$  Where $\sigma_x (n)$ 
	denotes the sum of positive divisors function.  The study of such numbers dates back to at least the 
	time of Euclid. In Book IX Proposition 36 his elements he details a elementary method for finding even
	perfect numbers.  This method was later labeled the Euclid–Euler theorem
	after Euler proved that the method produced even perfect numbers.
	
	The following is a proof of the afterfore mentioned theorem.
	
	\begin{theorem}
		An even natural number is perfect iff it of the form $$2^{p-1} (2^{p} - 1)$$
	\end{theorem}
	
	\begin{proof}
		$(\Rightarrow)$
			Show an even natural number is perfect if it is of the form  $$2^{p-1} (2^{p} - 1).$$
			Let $p \in \mathbb{N}$ s.t. $2^{p}-1$ is prime and $\sigma_x (n)$, as above, denote the sum of positive divisors function.
			It follows that $$\sigma (2^{p-1}(2^{p}-1))=\sigma (2^{p-1})\sigma (2^{p}-1) \newline =>
			\sigma (2^{p-1}(2^{p}-1))=\sigma (2^{p-1})\sigma (2^{p}-1)$$
			Where $\sigma (2^{p-1}) = \sum{}{}(1 + 2 + 4 + ... + 2^{p-1})$ is a geometric series.\newline
			Thus $\sigma (2^{p-1}) = 2^{p}-1)$ and $\sigma (2^{p}-1) = 2^{p}$ since $2^{p}-1$ is prime.
		
		$(\Leftarrow)$
			Show if an even natural number is of the form  $$2^{p-1} (2^{p} - 1)$$ then it is perfect.
			Suppose $n \in \mathbb{N}$ s.t. $$n = 2^{p-1} (2^{p} - 1)$$ for $p \in \mathbb{N}$
			Show n is perfect.
			Cosider the partial factorization of $n = 2^{x}y$ where y is odd.
			If $2^{x}$ is perfect then 2^{x}y = \sigma (2^{x}y) = 
 
	\end{proof}
	
\paragraph*{History of the search}
	
\paragraph*{Conclusion}

\paragraph*{Acknowledgment}


\begin{thebibliography}{5}
	\bibitem{lowerbound} 
		\textit{Ochem, P. and Rao, M. "Odd Perfect Numbers Are Greater than 10\^(15000)."} Math. Comput. 81, 1869-1877, 2012.
	\bibitem{elements}
		\textit{“Proposition 36.” Euclid's Elements, Book IX, Proposition 36,} https://mathcs.clarku.edu/~djoyce/java/elements/bookIX/propIX36.html.
\end{thebibliography}

\end{document}
